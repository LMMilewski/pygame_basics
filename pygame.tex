\documentclass{beamer}
\usetheme{Warsaw}
%\usecolortheme{seahorse}
\usepackage{graphicx}
\usepackage[utf8]{inputenc}
\usepackage[T1]{fontenc}
\title{Programowanie gier w PyGame}
\author{Łukasz Milewski}
\institute{Uniwersytet Wrocławski}
\date{\today, Wrocław}

\begin{document}

\begin{frame}
  \titlepage
\end{frame}

\begin{frame}
  \tableofcontents
\end{frame}

\section{Intro}

\begin{frame}{Target}

  \begin{block}{Dla kogo jest ten wykład?}
    \begin{itemize}
    \item Początkujący programiści
    \item Osoby chcę tworzyć gry dla zabawy
    \item Osoby oczekujące 'szybkich' efektów
    \item Chętni do uczenia się nowych rzeczy
    \end{itemize}
  \end{block}

  \begin{alertblock}{Dla kogo to nie jest wykład?}
    \begin{itemize}
    \item Nastawieni na komercyjny gamedev
    \item Zaawansowani programiści
    \item Osoby chcące tworzyć gry na konsole
    \end{itemize}
  \end{alertblock}

\end{frame}


\section{Ogólnie o tworzeniu gier}
\begin{frame}{OOP, DOD, CBP i inne literki}
  Gra amatorska to nie komercyjne oprogramowanie

  \begin{alertblock}{Czego nie warto robić}
    \begin{itemize}
    \item OOP, DOD, CBP, TDD i inne literki
    \item Clean code
    \item Pisanie własnego frameworka, enginu, czegokolwiek
    \item Uogólnianie
    \end{itemize}
  \end{alertblock}

  \begin{block}{Co warto robić}
    \begin{itemize}
    \item Pisać grę
    \item Dodawać kolejne featury
    \item Stosować KISS
    \item Maksimum efektów przy minimalnym nakładzie pracy
    \end{itemize}
  \end{block}

\end{frame}

\begin{frame}{System kontroli wersji}
  \begin{block}{github}
    \begin{itemize}
    \item https://github.com/
    \item http://help.github.com/create-a-repo/
    \end{itemize}
  \end{block}

  \begin{block}{git}
    \begin{itemize}
    \item git clone
    \item git push origin master - za pierwszym razem
    \item git commit -a -m 'tutaj opis zmian'
    \item git push
    \item gitk
    \end{itemize}
  \end{block}
\end{frame}

\begin{frame}{System kontroli wersji}
  \begin{block}{git a paraca grupowa}
    \begin{itemize}
    \item git remote add NAZWA ADRES
    \item git fetch NAZWA
    \item git merge NAZWA/master  (integrator)
    \item git rebase NAZWA/master (collaborator)
    \end{itemize}
  \end{block}
\end{frame}

\begin{frame}{Motywacja}
  \begin{itemize}
  \item Programowanie z kimś jeszcze (jak wybrać taką osobę?)
  \item Konkursy (np. www.pyweek.org, compo)
  \end{itemize}
\end{frame}

\begin{frame}{Dwie najskuteczniejsze metody nauczenia się progrmowania gier}

  \begin{block}{Pisz gry}
    Praktyka czyni mistrza. Co więcej - każda, nawet najprostsza,
    skończona gra daje sporą dawkę motywacji do tworzenia
    kolejnych. Uczenie się z tutoriali jest nieefektywne.
  \end{block}

  \begin{block}{Czytaj kod}
    Na www.pyweek.org oraz www.pygame.org jest bardzo wiele
    przykładowych gier. Warto wybierać te, które wygrywały poprzednie
    edycje pyweeka i zobaczyć jak są zrobione. Kod zazwyczaj nie jest
    piękny, obiektowy czy zgodny z inną ideologią. Za to działa, jest
    skończony, rozwiązuje konkretne problemy i ma więcej featurów niż
    pozostałe gry w danej edycji.
  \end{block}

\end{frame}


\section{Podstawy języka Python}
\begin{frame}{Dlaczego warto wybrać Python}
  \begin{block}{Zalety}
    \begin{itemize}
    \item Pygame
    \item Bardzo łatwy do nauczenia się
    \item REPL (toplevel)
    \item Można zrobić kompletną grę w tydzień
    \item Dużo bibliotek
    \item Bogata biblioteka standardowa (np. moduł random)
    \end{itemize}
  \end{block}
\end{frame}

\begin{frame}{Wady Pythona}
  \begin{block}{Wady}
    \begin{itemize}
    \item Wydajność
    \item Wydajność
    \item Wydajność
    \end{itemize}
  \end{block}
\end{frame}

\begin{frame}{Python - składnia}
  \begin{block}{liczby, booleans, słowniki, krotki, if, listy, for, range}
    Zobacz data\_types.py
  \end{block}

  \begin{block}{slicing, list comprehensions}
    Zobacz lists.py
  \end{block}

  \begin{block}{funkcje, klasy, metody}
    Zobacz fun\_class.py
  \end{block}
\end{frame}

\begin{frame}{Python - debugging}
  \begin{block}{Jak wyszukiwać błędy?}
    \begin{itemize}
    \item użyj debuggera [restart, p, c, b] (prezentacja)
    \item print debugging (moduł pprint)
    \item bisekcja
    \end{itemize}
  \end{block}
\end{frame}

\begin{frame}{Python - moduły}
  \begin{block}{sys}
    \begin{itemize}
    \item import sys
    \item sys.argv (sound = not "--nosound" in sys.argv)
    \item sys.argv[0]
    \end{itemize}
  \end{block}

  \begin{block}{os}
    \begin{itemize}
    \item import os
    \item base\_path = os.path.abspath(os.path.dirname(sys.argv[0]))
    \item os.path.* (join, abspath, dirname, isfile)
    \end{itemize}
  \end{block}

\end{frame}


\section{Tworzymy grę}

\begin{frame}{Struktura katalogów}
  \begin{block}{Płaska struktura katalogów}
    \begin{enumerate}
    \item font/
    \item gfx/
    \item music/
    \item sounds/
    \item src/
    \end{enumerate}
  \end{block}

  \begin{alertblock}{Pliki źródłowe}
    Warto trzymać wszystkie pliki źródłowe - np. pliki .xcf, .psd itp.
  \end{alertblock}
\end{frame}

\begin{frame}{Ważne pliki}
  \begin{block}{Źródła}
    \begin{enumerate}
    \item const.py
    \item config.py
    \item sprite.py
    \item resources.py
    \item main.py
    \end{enumerate}
  \end{block}

  \begin{alertblock}{}
    Dobrze jest zrobić te obiekty jako zmienne globalne (słowo
    kluczowe \alert{global})
  \end{alertblock}
\end{frame}

\begin{frame}{Podstawy}
  \begin{block}{Klasa Game}
    \begin{enumerate}
    \item def init(self, screen)
    \item def update(self, dt)
    \item def display(self, screen)
    \item def process\_event(self, event)
    \item self.is\_finished
    \end{enumerate}
  \end{block}

  \begin{block}{Kontekst}
    W konstruktorze tworzymy obiekty gry oraz sprite'y. Do obiektów
    dobrze jest przekazać obiekt Game (self)
  \end{block}
\end{frame}

\begin{frame}[fragile]{Podstawy}
  \begin{exampleblock}{Pętla główna}
    \alert{main\_loop.py}
  \end{exampleblock}

  \begin{block}{TBA}
\begin{verbatim}
  speed = self.current\_speed()
  px, py = self.position
  dx, dy = self.direction
  px, py = px + dx * dt * speed, py + dy * dt * speed
\end{verbatim}
  \end{block}
\end{frame}

\begin{frame}{Ładowanie zasobów}
  \begin{block}{Zasoby}
    \begin{itemize}
    \item pygame.image.load("obrazek.png").convert\_alpha()
    \item sound = pygame.mixer.Sound("dzwiek.ogg")
    \item music - jest streaming. nie trzeba niczego ładować
    \item pygame.font.Font("font.ttf", size)
    \item \alert{load\_animation.py}
    \end{itemize}
  \end{block}
\end{frame}

\begin{frame}{Wykorzystanie zasobów}
  \begin{block}{Zasoby}
    \begin{itemize}
    \item screen.blit(img, position)
    \item sound.play(loop)
    \item pygame.mixer.music.load(name), pygame.mixer.music.play(repeat)
    \item text\_img = font.render(text, 1, color)
    \end{itemize}
  \end{block}
\end{frame}

\begin{frame}{Przetwarzanie wejścia/wyjścia}
  \begin{block}{}
    \begin{itemize}
    \item polling
    \item zdarzenia \alert{events.py}
    \end{itemize}
  \end{block}
\end{frame}

\begin{frame}[fragile]{Kolizje}
  \begin{exampleblock}{Kolizje z główną postacią}
\begin{verbatim}
  for enemy in self.enemies:
  if aabb_collision(self.ferris.aabb(), enemy.aabb()):
  self.die()
  return
\end{verbatim}
  \end{exampleblock}

  \begin{exampleblock}{kolizje - odległość}
\begin{verbatim}
  def distance(pos1, pos2):
  dx = pos1[0] - pos2[0]
  dy = pos1[1] - pos2[1]
  return math.sqrt(dx*dx + dy*dy)

  if distance(player, register) < 5:
  player.pick(register)
\end{verbatim}
  \end{exampleblock}
\end{frame}

\begin{frame}{Kolizje}
  \begin{exampleblock}{kolizje AABB}
    def aabb\_collision((minx1, miny1, maxx1, maxy1), (minx2, miny2, maxx2, maxy2)):
      xcollision = (minx1 <= minx2 and minx2 <= maxx1) or (minx2 <= minx1 and minx1 <= maxx2)
      ycollision = (miny1 <= miny2 and miny2 <= maxy1) or (miny2 <= miny1 and miny1 <= maxy2)
      return xcollision and ycollision
  \end{exampleblock}

  \begin{block}{Pixel perfect}
    Bardzo dokładne kolizje, jednak w Pythonie trudno jest zrealizować
    z powodu dużego kosztu obliczeniowego. Dlatego odradzam ich
    stosowanie.
  \end{block}

\end{frame}


\begin{frame}{Kolizje}
  \begin{block}{Gdzie wykrywać kolizje?}
      \begin{itemize}
      \item W klasie Player
      \item W klasie Game
      \end{itemize}
  \end{block}
\end{frame}

\section{Najważniejsze moduły pygame}

\begin{frame}{pygame.surface}
  \begin{block}{}
    \begin{itemize}
    \item blit
    \item convert, convert\_alpha
    \item copy
    \item set\_at
    \item get\_at
    \end{itemize}
  \end{block}
\end{frame}

\begin{frame}{pygame.draw}
  \begin{block}{}
    \begin{itemize}
    \item rect
    \item polygon
    \item circle
    \item ellipse
    \item arc
    \item line
    \item lines
    \item aaline
    \item aalines
    \end{itemize}
  \end{block}
\end{frame}

\begin{frame}{pygame.Rect}
  \begin{block}{}
    \begin{itemize}
    \item copy
    \item contains
    \item collidepoint
    \item colliderect
    \item collidelist
    \item collidelistall
    \item collidedict
    \item collidedictall
    \end{itemize}
  \end{block}
\end{frame}

\begin{frame}{pygame.transform}
  \begin{block}{}
    \begin{itemize}
    \item flip
    \item scale
    \item rotate
    \item laplacian
    \item average\_surfaces
    \item average\_color
    \end{itemize}
  \end{block}
\end{frame}

\begin{frame}{inne}
  \begin{block}{}
    \begin{itemize}
    \item pygame.display.set\_caption('Title')
    \item pygame.mouse.set\_visible(True/False)
    \item pygame.sprite
    \item ...
    \end{itemize}
  \end{block}
\end{frame}

\begin{frame}{Przydatne adresy}
  \begin{block}{Linki}
    \begin{itemize}
    \item http://www.python.org/doc/
    \item www.pygame.org
    \item http://www.pygame.org/docs/ref/
    \item www.pyweek.org
    \item http://github.com
    \item https://github.com/lmmilewski/pygame\_basics
    \end{itemize}
  \end{block}

\end{frame}

\end{document}
