\documentclass{beamer}
\usetheme{Warsaw}
%\usecolortheme{seahorse}
\usepackage{graphicx}
\usepackage[utf8]{inputenc}
\usepackage[T1]{fontenc}
\title{Programowanie gier w PyGame}
\author{Łukasz Milewski}
\institute{Uniwersytet Wrocławski}
\date{\today, Wrocław}

\begin{document}

\begin{frame}
  \titlepage
\end{frame}

\section{Intro}

\begin{frame}{Target}

  \begin{block}{Dla kogo jest ten wykład?}
    \begin{itemize}
    \item Początkujący programiści
    \item Osoby chcę tworzyć gry dla zabawy
    \item Osoby oczekujące 'szybkich' efektów
    \item Chętni do uczenia się nowych rzeczy
    \end{itemize}
  \end{block}

  \begin{alertblock}{Dla kogo to nie jest wykład?}
    \begin{itemize}
    \item Nastawieni na komercyjny gamedev
    \item Zaawansowani programiści
    \item Osoby chcące tworzyć gry na konsole
    \end{itemize}
  \end{alertblock}

\end{frame}


\section{Ogólnie o tworzeniu gier}
\begin{frame}{OOP, DOD, CBP i inne literki}
  Gra amatorska to nie komercyjne oprogramowanie

  \begin{alertblock}{Czego nie warto robić}
    \begin{itemize}
    \item OOP, DOD, CBP, TDD i inne literki
    \item Clean code
    \item Pisanie własnego frameworka, enginu, czegokolwiek
    \item Uogólnianie
    \end{itemize}
  \end{alertblock}

  \begin{block}{Co warto robić}
    \begin{itemize}
    \item Pisać grę
    \item Dodawać kolejne featury
    \item Stosować KISS
    \item Maksimum efektów przy minimalnym nakładzie pracy
    \end{itemize}
  \end{block}

\end{frame}

\begin{frame}{System kontroli wersji}
  \begin{block}{github}
    \begin{itemize}
    \item https://github.com/
    \item http://help.github.com/create-a-repo/
    \end{itemize}
  \end{block}

  \begin{block}{git}
    \begin{itemize}
    \item git clone
    \item git push origin master - za pierwszym razem
    \item git commit -a -m 'tutaj opis zmian'
    \item git push
    \item gitk
    \end{itemize}
  \end{block}
\end{frame}

\begin{frame}{System kontroli wersji}
  \begin{block}{git a paraca grupowa}
    \begin{itemize}
    \item git remote add NAZWA ADRES
    \item git fetch NAZWA
    \item git merge NAZWA/master  (integrator)
    \item git rebase NAZWA/master (collaborator)
    \end{itemize}
  \end{block}
\end{frame}

\begin{frame}{Motywacja}
  \begin{itemize}
  \item Programowanie z kimś jeszcze (jak wybrać taką osobę?)
  \item Konkursy (np. www.pyweek.org, compo)
  \end{itemize}
\end{frame}

\begin{frame}{Dwie najskuteczniejsze metody nauczenia się progrmowania gier}

  \begin{block}{Pisz gry}
    Praktyka czyni mistrza. Co więcej - każda, nawet najprostsza,
    skończona gra daje sporą dawkę motywacji do tworzenia
    kolejnych. Najwięcej można się nauczyć rozwiązując konkretne
    problemy z konkretną grą. Uczenie się z tutoriali (lub co gorsze -
    z książek) jest bardzo nieefektywne.
  \end{block}

  \begin{block}{Czytaj kod}
    Na www.pyweek.org oraz www.pygame.org jest bardzo wiele
    przykładowych gier. Warto wybierać np. te, które wygrywały
    poprzednie edycje pyweeka i zobaczyć jak są zrobione. Kod
    zazwyczaj nie jest piękny, obiektowy czy zgodny z inną
    ideologią. Za to działa, jest skończony, rozwiązuje konkretne
    problemy i ma więcej featurów niż pozostałe gry w danej edycji.
  \end{block}

\end{frame}


\section{Podstawy języka Python}
\begin{frame}{Dlaczego warto wybrać Python}
  \begin{block}
    \begin{itemize}
    \item Pygame
    \item Bardzo łatwy do nauczenia się
    \item REPL (toplevel)
    \item Można zrobić kompletną grę w tydzień
    \item Dużo bibliotek
    \item Bogata biblioteka standardowa (np. moduł random)
    \end{itemize}
  \end{block}
\end{frame}

\begin{frame}{Wady Pythona}
  \begin{block}
    \begin{itemize}
    \item Wydajność
    \item Wydajność
    \item Wydajność
    \end{itemize}
  \end{block}
\end{frame}

\begin{frame}{Python - składnia}
  \begin{block}{liczby, booleans, słowniki, krotki, if, listy, for, range}
    Zobacz data\_types.py
  \end{block}

  \begin{block}{slicing, list comprehensions}
    Zobacz lists.py
  \end{block}

  \begin{block}{funkcje, klasy, metody}
    Zobaczk fun\_class.py
  \end{block}
\end{frame}

\begin{frame}{Python - debugging}
  \begin{block}{Jak wyszukiwać błędy?}
    \begin{itemize}
    \item użyj debuggera [restart, p, c, b] (prezentacja)
    \item print debugging (moduł pprint)
    \item bisekcja
    \end{itemize}
  \end{block}
\end{frame}

\begin{frame}{Python - moduły}
  \begin{block}{sys}
    \begin{itemize}
    \item import sys
    \item sys.argv (sound == not "--nosound" in sys.argv)
    \item sys.argv[0]
    \end{itemize}
  \end{block}

  \begin{block}{os}
    \begin{itemize}
    \item import os
    \item base\_path = os.path.abspath(os.path.dirname(sys.argv[0]))
    \item os.path.* (join, abspath, dirname, isfile)
    \end{itemize}
  \end{block}

\end{frame}


\section{Tworzymy grę}
% const.py
% config.py
% Pygame nadaje się tylko do prostych gier 2D (np. tetris, pong,
% pacman, ...)
% Proponowana struktura katalogów
% Pętla główna
% Time based animation
% Klasa Game
% Singletony / zmienne globalne
% Lista obiektów
% Processing events (state polling, event driven)
% Displaying
% Fonts
% Sound
% Music
% pygame.display.set_caption('Title')
% pygame.mouse.set_visible(True/False)
% Sprites (sprite.py)
% Obrazki
% Config.py
% pygame.transform
% pygame.surface (fill, blit, set_at, get_at)
% convert_alpha()
% many files for animation / texture atlas?
% AABB i kolizje, pixel perfect collisions?

\end{document}
